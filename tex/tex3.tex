\documentclass[dvipdfmx]{jsarticle}

\title{スッキリわかるサーブレット\&JSP入門 第13章を \\ MySQLを使ってやってみる \\ Ver1.5}
\author{Seiichi Nukayama}
\date{\today}
\usepackage{tcolorbox}
\usepackage{color}
\usepackage{listings, plistings}

% Java
\lstset{% 
  frame=single,
  backgroundcolor={\color[gray]{.9}},
  stringstyle={\ttfamily \color[rgb]{0,0,1}},
  commentstyle={\itshape \color[cmyk]{1,0,1,0}},
  identifierstyle={\ttfamily}, 
  keywordstyle={\ttfamily \color[cmyk]{0,1,0,0}},
  basicstyle={\ttfamily},
  breaklines=true,
  xleftmargin=0zw,
  xrightmargin=0zw,
  framerule=.2pt,
  columns=[l]{fullflexible},
  numbers=left,
  stepnumber=1,
  numberstyle={\scriptsize},
  numbersep=1em,
  language={Java},
  lineskip=-0.5zw,
  morecomment={[s][{\color[cmyk]{1,0,0,0}}]{/**}{*/}},
}
%\usepackage[dvipdfmx]{graphicx}
\usepackage{url}
\usepackage[dvipdfmx]{hyperref}
\usepackage{amsmath, amssymb}
\usepackage{itembkbx}
\usepackage{eclbkbox}	% required for `\breakbox' (yatex added)
\usepackage{enumerate}
\fboxrule=0.5pt
\parindent=1em
\begin{document}

%\anaumeと入力すると穴埋め解答欄が作れるようにしてる。\anaumesmallで小さめの穴埋めになる。
\newcounter{mycounter} % カウンターを作る
\setcounter{mycounter}{0} % カウンターを初期化
\newcommand{\anaume}[1][]{\refstepcounter{mycounter}{#1}{\boxed{\phantom{aa}\themycounter \phantom{aa}}}} %穴埋め問題の空欄作ってる。
\newcommand{\anaumesmall}[1][]{\refstepcounter{mycounter}{#1}{\boxed{\tiny{\phantom{a}\themycounter \phantom{a}}}}}%小さい版作ってる。色々改造できる。

%% 修正時刻: Sat 2024/01/13 05:56:141


\section{サンプルプログラム(p383)の作成}

p383に掲載されているサンプルプログラムのMySQL版を書いてみる。

仮にこの本のプログラムコードを入れているフォルダを ''sukkiri'' とする。
この中に ''chap13-mysql'' というフォルダを作成する。
このフォルダの中に入り、''terminal'' というフォルダを作成する。
この中に更に ''src'' と ''classes'' と ''lib'' というフォルダを作成する。

\begin{tcolorbox}
\begin{verbatim}
chap13-mysql/
└── terminal/
    ├── classes/
    ├── lib/
    └── src/
\end{verbatim}
\end{tcolorbox}

''chap13-mysql''フォルダには、この章で作成するサーブレットJSPファイルを置く。 \\
''terminal/src''フォルダには、p383のコードを置く。 \\
''terminal/classes''フォルダには、p383のコードをコンパイルしてできたclassファイルを置く。 \\
''terminal/lib'' フォルダには、mysql-connector-java-8.0.21.jaqを置く。 \\

''src''フォルダで以下のコードを書く。

\begin{lstlisting}[caption=src/SelectEmployeeSample.java]
import java.sql.Connection;
import java.sql.DriverManager;
import java.sql.PreparedStatement;
import java.sql.ResultSet;
import java.sql.SQLException;

public class SelectEmployeeSample {

  static final String USERNAME = "sa";
  static final String PASSWORD = "sa";
  static final String CONNECT =
               "jdbc:mysql://localhost:3306/example?serverTimezone=JST";
 
  public static void main( String[] args ) {
    // データベースに接続
    try (Connection conn =
       DriverManager.getConnection( CONNECT, USERNAME, PASSWORD )) {
      // select 文
      String sql = "select id, name, age from employee";
      PreparedStatement pStmt = conn.prepareStatement( sql );

      ResultSet rs = pStmt.executeQuery();

      while( rs.next() ) {
        String id = rs.getString("id");
        String name = rs.getString("name");
        int age = rs.getInt("age");

        System.out.println("ID:" + id);
        System.out.println("名前:" + name);
        System.out.println("年齢:" + age + "\n");
      }
    } catch (SQLException e) {
      e.printStackTrace();
    }
  }
}
\end{lstlisting}

``?serverTimezone=JST'' は、これを書かないと、Timezoneが設定されていないというエラーが出る場合がある(XAMPPの場合?)。


''terminal''フォルダにて、次の build.xml を書く。

\begin{lstlisting}[caption=build.xml]
 <?xml version="1.0" ?>
<project name="sample" default="compile" basedir=".">
  <target name="compile">
    <javac includeAntRuntime="false"
           srcdir="./src"
           destdir="./classes"
           classpath="./lib/mysql-connector-java-8.0.21.jar"
           />
  </target>
</project>
\end{lstlisting}

コンパイルする。

\begin{tcolorbox}
\begin{verbatim}
> ant
compile:

BUILD SUCCESSFUL
\end{verbatim}
\end{tcolorbox}


実行する。

\begin{tcolorbox}
\begin{verbatim}
> cd classes
> java -cp .;../lib/mysql-connector-java-8.0.20.jar SelectEmployeeSample
\end{verbatim}
\end{tcolorbox}

-cp オプションで、現在のフォルダ「.」と「lib」フォルダのjarファイルを指定している。\\
mysql-connector-java-8.0.21.jar は、コンパイル時にも、実行時にも、-classpath指定が必要である。

\begin{verbatim}
ID:EMP001
名前:湊 雄輔
年齢:23

ID:EMP002
名前:綾部 みゆき
年齢:22
\end{verbatim}

しかし、毎回このオプションを入力するのはとても邪魔くさいので、バッチファイルを作成して、
ちょっとでも楽をする。

class フォルダにて、``\textsf{run.bat}''というファイルを作成し、内容を以下とする。

\begin{lstlisting}
 java -cp .;../lib/mysql-connector-java-8.0.21.jar %1
\end{lstlisting}

そして、classフォルダにてコマンドプロンプトを起動し、以下のようにする。

\begin{tcolorbox}
 \textgreater run SelectEmployeeSample
\end{tcolorbox}


\end{document}

%% 修正時刻: Sat May  2 15:10:04 2020


%% 修正時刻: Mon Jul 27 06:45:15 2020
