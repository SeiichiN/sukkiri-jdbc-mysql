\documentclass[dvipdfmx]{jsarticle}

\title{スッキリわかるサーブレット\&JSP入門 第13章を \\ MySQLを使ってやってみる \\ Ver1.5}
\author{Seiichi Nukayama}
\date{\today}
\usepackage{tcolorbox}
\usepackage{color}
\usepackage{listings, plistings}

% Java
\lstset{% 
  frame=single,
  backgroundcolor={\color[gray]{.9}},
  stringstyle={\ttfamily \color[rgb]{0,0,1}},
  commentstyle={\itshape \color[cmyk]{1,0,1,0}},
  identifierstyle={\ttfamily}, 
  keywordstyle={\ttfamily \color[cmyk]{0,1,0,0}},
  basicstyle={\ttfamily},
  breaklines=true,
  xleftmargin=0zw,
  xrightmargin=0zw,
  framerule=.2pt,
  columns=[l]{fullflexible},
  numbers=left,
  stepnumber=1,
  numberstyle={\scriptsize},
  numbersep=1em,
  language={Java},
  lineskip=-0.5zw,
  morecomment={[s][{\color[cmyk]{1,0,0,0}}]{/**}{*/}},
}
%\usepackage[dvipdfmx]{graphicx}
\usepackage{url}
\usepackage[dvipdfmx]{hyperref}
\usepackage{amsmath, amssymb}
\usepackage{itembkbx}
\usepackage{eclbkbox}	% required for `\breakbox' (yatex added)
\usepackage{enumerate}
\fboxrule=0.5pt
\parindent=1em
\begin{document}

%\anaumeと入力すると穴埋め解答欄が作れるようにしてる。\anaumesmallで小さめの穴埋めになる。
\newcounter{mycounter} % カウンターを作る
\setcounter{mycounter}{0} % カウンターを初期化
\newcommand{\anaume}[1][]{\refstepcounter{mycounter}{#1}{\boxed{\phantom{aa}\themycounter \phantom{aa}}}} %穴埋め問題の空欄作ってる。
\newcommand{\anaumesmall}[1][]{\refstepcounter{mycounter}{#1}{\boxed{\tiny{\phantom{a}\themycounter \phantom{a}}}}}%小さい版作ってる。色々改造できる。

%% 修正時刻: Sat 2024/01/13 05:56:141


\section{サンプルプログラム(p383)の作成}

p383に掲載されているサンプルプログラムのMySQL版を書いてみる。

新規動的Webプロジェクトを作成する。プロジェクト名は ``\textsf{mysql}''。

''src''フォルダで以下のコードを書く。
パッケージを作るなら ''terminal'' とでもしておく。

\begin{lstlisting}[caption=src/SelectEmployeeSample.java]
import java.sql.Connection;
import java.sql.DriverManager;
import java.sql.PreparedStatement;
import java.sql.ResultSet;
import java.sql.SQLException;

public class SelectEmployeeSample {

  static final String USERNAME = "sa";
  static final String PASSWORD = "";
  static final String CONNECT =
               "jdbc:mysql://localhost:3306/example";
 
  public static void main( String[] args ) {
    // MySQLドライバーをDriverManagerに登録
    try {
      Class.forName("com.mysql.cj.jdbc.Driver");
    } catch (ClassNotFoundException e) {
      throw new IllegalStateException("ドライバーが見つかりません");
    }
    // データベースに接続
    try (Connection conn =
       DriverManager.getConnection( CONNECT, USERNAME, PASSWORD )) {
      // select 文
      String sql = "SELECT id, name, age FROM employee";
      PreparedStatement pStmt = conn.prepareStatement( sql );
      ResultSet rs = pStmt.executeQuery();

      while( rs.next() ) {
        String id = rs.getString("id");
        String name = rs.getString("name");
        int age = rs.getInt("age");

        System.out.println("ID:" + id);
        System.out.println("名前:" + name);
        System.out.println("年齢:" + age + "\n");
      }
    } catch (SQLException e) {
      e.printStackTrace();
    }
  }
}
\end{lstlisting}

Timezoneが設定されていないというエラーが出る場合、以下のようにする。\\
  static final String CONNECT =
               "jdbc:mysql://localhost:3306/example?serverTimezone=JST";


これを 実行 -- Javaアプリケーション とすると、以下のように出力される。
               
\begin{verbatim}
ID:EMP001
名前:湊 雄輔
年齢:23

ID:EMP002
名前:綾部 みゆき
年齢:22
\end{verbatim}



\end{document}

%% 修正時刻: Sat May  2 15:10:04 2020


%% 修正時刻: Sat 2023/12/30 06:09:441
