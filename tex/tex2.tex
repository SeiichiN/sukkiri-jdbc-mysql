\documentclass[dvipdfmx]{jsarticle}

\title{スッキリわかるサーブレット\&JSP入門 第13章をMySQLを使ってやってみる}
\author{Seiichi Nukayama}
\date{2020-07-26}
\usepackage{tcolorbox}
\usepackage{color}
\usepackage{listings, plistings}

% Java
\lstset{% 
  frame=single,
  backgroundcolor={\color[gray]{.9}},
  stringstyle={\ttfamily \color[rgb]{0,0,1}},
  commentstyle={\itshape \color[cmyk]{1,0,1,0}},
  identifierstyle={\ttfamily}, 
  keywordstyle={\ttfamily \color[cmyk]{0,1,0,0}},
  basicstyle={\ttfamily},
  breaklines=true,
  xleftmargin=0zw,
  xrightmargin=0zw,
  framerule=.2pt,
  columns=[l]{fullflexible},
  numbers=left,
  stepnumber=1,
  numberstyle={\scriptsize},
  numbersep=1em,
  language={Java},
  lineskip=-0.5zw,
  morecomment={[s][{\color[cmyk]{1,0,0,0}}]{/**}{*/}},
}
%\usepackage[dvipdfmx]{graphicx}
\usepackage{url}
\usepackage[dvipdfmx]{hyperref}
\usepackage{amsmath, amssymb}
\usepackage{itembkbx}
\usepackage{eclbkbox}	% required for `\breakbox' (yatex added)
\usepackage{enumerate}
\fboxrule=0.5pt
\parindent=1em
\begin{document}

%\anaumeと入力すると穴埋め解答欄が作れるようにしてる。\anaumesmallで小さめの穴埋めになる。
\newcounter{mycounter} % カウンターを作る
\setcounter{mycounter}{0} % カウンターを初期化
\newcommand{\anaume}[1][]{\refstepcounter{mycounter}{#1}{\boxed{\phantom{aa}\themycounter \phantom{aa}}}} %穴埋め問題の空欄作ってる。
\newcommand{\anaumesmall}[1][]{\refstepcounter{mycounter}{#1}{\boxed{\tiny{\phantom{a}\themycounter \phantom{a}}}}}%小さい版作ってる。色々改造できる。

%% 修正時刻: Tue May  5 10:19:29 2020


\section{データを準備する}

p383 のコードを入力する前に、このコードで使っているデータを \textsf{MySQL} で準備しておかなくてはならない。

\subsection{MySQL:ユーザーの作成}

XAMPPなどで、MySQLを動作させておく。

まず、MySQL で使用するユーザーを用意しなくてはならない。
ROOT のままだと、他のデータベースにもアクセス可能なので、実際にはそのデータベースにのみアクセス権限が与えられているユーザーを作成することになる。しかし今はお試しでプログラムを作成しているので、ユーザー名とパスワードは簡単なものにしておく。

\begin{tcolorbox}
 ユーザー名: \textsf{sa} \\
 パスワード: (なし)
\end{tcolorbox}

本(p383)ではパスワードは設定されていないので、同じようにしておく。

\begin{enumerate}
 \item MySQLにルートでログインする。\\
       Windows のコマンドプロンプトで、以下のようにする。\\
       \quad \verb! > mysql -u root -p ! \\
       \quad \verb! Password: **** ! \\
       (多くの場合、パスワードは設定されていないか、もしくはroot である) \\
 \item ユーザーを作成する。\\
       \quad \verb! mysql> create user 'sa'@'localhost' identified by ''; ! \\
       (ユーザー''sa''を作成し、パスワードを''''としている) \\
 \item そのユーザーにこれから作成するデータベースへの権限を与える。 \\
       \quad \verb! mysql> grant all on example.* to 'sa'@'localhost'; ! \\
       (データベース名は ``example'' で、それに関連する全てのファイルにアクセス権を与える) \\
 \item  MySQLをログアウトする \\
       \quad  \verb! mysql> quit; !
\end{enumerate}

以下のようにすると、ユーザーの作成とデータベースへの権限付与は同時に行うことができる。

\quad \verb! mysql> grant all on example.* 'sa'@'localhost' identified by ''; !



\subsection{データベースの作成}

先ほど作成したユーザーで MySQL にログインする。

\begin{tcolorbox}
\verb! > mysql -u sa -p ! \\
\verb!   Password: !   (そのままEnter)
\end{tcolorbox}

データベース(example)を作成する。

\begin{tcolorbox}
\verb! mysql> create database example ; !
\end{tcolorbox}

テーブル(employee)を作成する。

\begin{tcolorbox}
\begin{verbatim}
mysql> create table employees ( 
    -> id char(6) primary key, 
    -> name varchar(100) not null, 
    -> age int not null) ;
\end{verbatim}
\end{tcolorbox}

\subsection{サンプルデータを入れる}

サンプルデータを入れる。
データベースアプリを作るときは、最初にサンプルデータを入れておくようにする。

\begin{tcolorbox}
\begin{verbatim}
mysql> insert into employees values ( 'EMP001', '湊 雄輔', 23); 
mysql> insert into employees values ( 'EMP002', '綾部 みゆき', 22) ;
\end{verbatim}
\end{tcolorbox}

確認する。

\begin{verbatim}
mysql> select * from employees;
+--------+----------------+------+
| id     | name           | age  |
+--------+----------------+------+
| EMP001 | 湊 雄輔         |   23 |
| EMP002 | 綾部 みゆき     |   22 |
+--------+----------------+------+
2 rows in set (0.00 sec)
\end{verbatim}

データがちゃんと入っている。
\footnote{Windowsでは文字コードが Shift-JIS なので、漢字がうまく表示できない場合がある。
そのときは、とりあえずローマ字など英字で入れておく。プログラムを実行したときに
入力に漢字(UTF-8)が使えればよい。}



\include{end}

%% 修正時刻: Sat 2024/01/13 05:56:141

