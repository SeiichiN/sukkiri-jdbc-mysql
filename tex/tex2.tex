\documentclass[dvipdfmx]{jsarticle}

\title{スッキリわかるサーブレット\&JSP入門 第13章をMySQLを使ってやってみる}
\author{Seiichi Nukayama}
\date{2020-07-26}
\usepackage{tcolorbox}
\usepackage{color}
\usepackage{listings, plistings}

% Java
\lstset{% 
  frame=single,
  backgroundcolor={\color[gray]{.9}},
  stringstyle={\ttfamily \color[rgb]{0,0,1}},
  commentstyle={\itshape \color[cmyk]{1,0,1,0}},
  identifierstyle={\ttfamily}, 
  keywordstyle={\ttfamily \color[cmyk]{0,1,0,0}},
  basicstyle={\ttfamily},
  breaklines=true,
  xleftmargin=0zw,
  xrightmargin=0zw,
  framerule=.2pt,
  columns=[l]{fullflexible},
  numbers=left,
  stepnumber=1,
  numberstyle={\scriptsize},
  numbersep=1em,
  language={Java},
  lineskip=-0.5zw,
  morecomment={[s][{\color[cmyk]{1,0,0,0}}]{/**}{*/}},
}
%\usepackage[dvipdfmx]{graphicx}
\usepackage{url}
\usepackage[dvipdfmx]{hyperref}
\usepackage{amsmath, amssymb}
\usepackage{itembkbx}
\usepackage{eclbkbox}	% required for `\breakbox' (yatex added)
\usepackage{enumerate}
\fboxrule=0.5pt
\parindent=1em
\begin{document}

%\anaumeと入力すると穴埋め解答欄が作れるようにしてる。\anaumesmallで小さめの穴埋めになる。
\newcounter{mycounter} % カウンターを作る
\setcounter{mycounter}{0} % カウンターを初期化
\newcommand{\anaume}[1][]{\refstepcounter{mycounter}{#1}{\boxed{\phantom{aa}\themycounter \phantom{aa}}}} %穴埋め問題の空欄作ってる。
\newcommand{\anaumesmall}[1][]{\refstepcounter{mycounter}{#1}{\boxed{\tiny{\phantom{a}\themycounter \phantom{a}}}}}%小さい版作ってる。色々改造できる。

%% 修正時刻: Tue May  5 10:19:29 2020


\section{データを準備する}

p383 のコードを入力する前に、このコードで使っているデータを \textsf{MySQL} で準備しておかなくてはならない。

\subsection{MySQL:ユーザーの作成}

XAMPPなどで、MySQLを動作させておく。

まず、MySQL で使用するユーザーを用意しなくてはならない。
ROOT のままだと、他のデータベースにもアクセス可能なので、実際にはそのデータベースにのみアクセス権限が与えられているユーザーを作成することになる。しかし今はお試しでプログラムを作成しているので、ユーザー名とパスワードは簡単なものにしておく。

\begin{tcolorbox}
 ユーザー名: \textsf{sa} \\
 パスワード: \textsf{sa}
\end{tcolorbox}

本(p383)ではパスワードは設定されていないが、ユーザー名と同じにしておいたら忘れることはないだろう。

\begin{enumerate}
 \item MySQLにルートでログインする。\\
       Windows のコマンドプロンプトで、以下のようにする。\\
       \verb! > mysql -u root -p ! \\
       \verb! Password: **** ! \\
       (多くの場合、パスワードは設定されていないか、もしくはroot である)  \\
 \item ユーザーを作成する。\\
       \verb! mysql> create user 'sa'@'localhost' identified by 'sa'; ! \\
       (ユーザー''sa''を作成し、パスワードを''sa''としている) 
       \footnote{\' シングルクォーテーションは無くてもいけるみたい。ただし、パスワードにはシングルクォーテーションはあった方がよい。シングルクォーテーションはそれが文字列であることを明示している。}
       \footnote{localhostの指定を無しにすることもできる。その場合、すべてのホストからそのデータベースに接続できることになる。多くの場合、あるホストで動くプログラムから接続するだけだし、そのプログラムとMySQLは同じホストで動作しているだろうから、ここは localhost という指定があったほうが、セキュリティ上好ましい。} \\
 \item そのユーザーにこれから作成するデータベースへの権限を与える。 \\
       \verb! mysql> grant all privileges on example.* to 'sa'@'localhost'; ! \\
       (データベース名は ``example'' で、それに関連する全てのファイルにアクセス権を与える) \\
 \item \verb! mysql> quit; ! (MySQLをログアウトする)
\end{enumerate}

以下のようにすると、ユーザーの作成とデータベースへの権限付与は同時に行うことができる。

\verb! mysql> grant all on example.* 'sa'@'localhost' identified by 'sa'; !

\subsection{MySQLの文字コードの確認}

一応、MySQLの文字コードがどういう設定になっているかを確認しておく。
インターネットではUTF-8を使うのだが、Windows では Shift-JIS(cp932) だからである。

以下は、XAMPP の場合である。XAMPPをインストールしたときの初期設定である。

\begin{verbatim}
mysql> show variables like '%char%'; \\
+--------------------------+--------------------------------+
| Variable_name            | Value                          |
+--------------------------+--------------------------------+
| character_set_client     | cp932                          |
| character_set_connection | cp932                          |
| character_set_database   | utf8mb4                        |
| character_set_filesystem | binary                         |
| character_set_results    | cp932                          |
| character_set_server     | utf8mb4                        |
| character_set_system     | utf8                           |
| character_sets_dir       | C:\xampp\mysql\share\charsets\ |
+--------------------------+--------------------------------+
8 rows in set (0.00 sec)
\end{verbatim}


\subsection{データベースの作成}

先ほど作成したユーザーで MySQL にログインする。

\begin{tcolorbox}
\verb! > mysql -u sa -p ! \\
\verb!   Password: ** !   (sa)
\end{tcolorbox}

データベース(example)を作成する。使用する文字コードも指定しておく。
\footnote{データベース作成後に文字コードを設定する場合は、
mysql\textgreater alter database example default character set=utf8; とする。}

\begin{tcolorbox}
\verb! mysql> create database example default character set=utf8; !
\end{tcolorbox}

テーブル(employee)を作成する。使用する文字コードも指定しておく。
\footnote{テーブル作成後に文字コードを設定する場合は、
mysql\textgreater alter table employee default character set=utf8; } 

\begin{tcolorbox}
\begin{verbatim}
mysql> create table employee ( 
    -> id char(6) not null primary key, 
    -> name varchar(40), 
    -> age int ) defalult character set=utf8; 
\end{verbatim}
\end{tcolorbox}

\subsection{サンプルデータを入れる}

サンプルデータを入れる。
データベースアプリを作るときは、最初にサンプルデータを入れておくようにする。

\begin{tcolorbox}
\begin{verbatim}
mysql> insert into employee values ( 'EMP001', '湊 雄輔', 23); 
mysql> insert into employee values ( 'EMP002', '綾部 みゆき', 22) ;
\end{verbatim}
\end{tcolorbox}

確認する。

\begin{verbatim}
mysql> select * from employee;
+--------+----------------+------+
| id     | name           | age  |
+--------+----------------+------+
| EMP001 | 湊 雄輔         |   23 |
| EMP002 | 綾部 みゆき     |   22 |
+--------+----------------+------+
2 rows in set (0.00 sec)
\end{verbatim}

データがちゃんと入っている。
\footnote{Windowsでは文字コードが Shift-JIS なので、漢字がうまく表示できない場合がある。
そのときは、とりあえずローマ字など英字で入れておく。プログラムを実行したときに
入力に漢字(UTF-8)が使えればよい。}



\include{end}

%% 修正時刻: Sun Jul 26 10:21:39 2020

