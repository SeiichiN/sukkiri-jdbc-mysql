\documentclass[dvipdfmx]{jsarticle}

\title{スッキリわかるサーブレット\&JSP入門 第13章を \\ MySQLを使ってやってみる \\ Ver1.5}
\author{Seiichi Nukayama}
\date{\today}
\usepackage{tcolorbox}
\usepackage{color}
\usepackage{listings, plistings}

% Java
\lstset{% 
  frame=single,
  backgroundcolor={\color[gray]{.9}},
  stringstyle={\ttfamily \color[rgb]{0,0,1}},
  commentstyle={\itshape \color[cmyk]{1,0,1,0}},
  identifierstyle={\ttfamily}, 
  keywordstyle={\ttfamily \color[cmyk]{0,1,0,0}},
  basicstyle={\ttfamily},
  breaklines=true,
  xleftmargin=0zw,
  xrightmargin=0zw,
  framerule=.2pt,
  columns=[l]{fullflexible},
  numbers=left,
  stepnumber=1,
  numberstyle={\scriptsize},
  numbersep=1em,
  language={Java},
  lineskip=-0.5zw,
  morecomment={[s][{\color[cmyk]{1,0,0,0}}]{/**}{*/}},
}
%\usepackage[dvipdfmx]{graphicx}
\usepackage{url}
\usepackage[dvipdfmx]{hyperref}
\usepackage{amsmath, amssymb}
\usepackage{itembkbx}
\usepackage{eclbkbox}	% required for `\breakbox' (yatex added)
\usepackage{enumerate}
\fboxrule=0.5pt
\parindent=1em
\begin{document}

%\anaumeと入力すると穴埋め解答欄が作れるようにしてる。\anaumesmallで小さめの穴埋めになる。
\newcounter{mycounter} % カウンターを作る
\setcounter{mycounter}{0} % カウンターを初期化
\newcommand{\anaume}[1][]{\refstepcounter{mycounter}{#1}{\boxed{\phantom{aa}\themycounter \phantom{aa}}}} %穴埋め問題の空欄作ってる。
\newcommand{\anaumesmall}[1][]{\refstepcounter{mycounter}{#1}{\boxed{\tiny{\phantom{a}\themycounter \phantom{a}}}}}%小さい版作ってる。色々改造できる。

%% 修正時刻: Sat 2024/01/13 05:56:141


\section{DAOパターンのサンプルプログラム}

本p390 からのサンプルプログラムを書いてみる。

\subsection{プロジェクトフォルダの設定}

先ほど作った ``chap13-mysql'' フォルダの中に ``dao'' というフォルダを作成する。

そして、以下のようにフォルダを作成する。

 \begin{tcolorbox}
  \begin{verbatim}
 dao
├── build.xml 
├── classes/ 
├── lib/ 
│   └── mysql-connector-java-8.0.21.jar 
└── src/ 
    ├── SelectEmployeeSample.java 
    ├── dao/ 
    │   └── EmployeeDAO.java 
    └── model/ 
         └── Employee.java 
\end{verbatim}
 \end{tcolorbox}

``build.xml''、``SelectEmployeeSample.java''、``EmployeeDAO.java''、``Employee.java'' は、これから作成するファイルである。

``mysql-connector-java-8.0.21.java'' は、先ほどダウンロードしたファイルである。


\subsection{プログラムコードを書く}

\begin{lstlisting}[caption=Employee.java]
 package model;

 public class Employee {
   private String id;
   private String name;
   private int age;
  
   public Employee( String id, String name, int age ) {
     this.id = id;
     this.name = name;
     this.age = age;
   }
  
   public String getId() { return id; }
   public String getName() { return name; }
   public int getAge() { return age; }
 }
\end{lstlisting}

\begin{lstlisting}[caption=EmployeeDAO.java]
package dao;

import java.sql.Connection;
import java.sql.DriverManager;
import java.sql.PreparedStatement;
import java.sql.ResultSet;
import java.sql.SQLException;
import java.util.ArrayList;
import java.util.List;

import model.Employee;

public class EmployeeDAO {
    private final String JDBC_URL =
        "jdbc:mysql://localhost:3306/example?serverTimezone=JST";
    private final String DB_USER = "sa";
    private final String DB_PASS = "sa";

    /**
     * 一覧表示メソッド
     * @Param: none
     * @Return:
     *   List<Employee> empList -- Employee型のリスト
     */
    public List<Employee> findAll () {
        List<Employee> empList = new ArrayList <> ();

        try (Connection conn =
             DriverManager.getConnection( JDBC_URL, DB_USER, DB_PASS )) {

            String sql = "select id, name, age from employee";
            PreparedStatement pStmt = conn.prepareStatement( sql );

            ResultSet rs = pStmt.executeQuery();

            while (rs.next()) {
                String id = rs.getString("id");
                String name = rs.getString("name");
                int age = rs.getInt("age");
                Employee employee = new Employee( id, name, age );
                empList.add( employee );
            }
        } catch (SQLException e) {
            e.printStackTrace();
            return null;
        }
        return empList;
    }

    /**
     * 削除処理
     * @Param:
     *   String id -- ex.EMP001
     * @Return:
     *   boolean -- true: 削除成功、  false: 削除失敗l
     */
    public boolean remove ( String id ) {
        try (Connection conn =
             DriverManager.getConnection( JDBC_URL, DB_USER, DB_PASS )) {

            String sql = "delete from employee where id = ?";
            PreparedStatement pStmt = conn.prepareStatement( sql );
            pStmt.setString(1, id);
            int result = pStmt.executeUpdate();
            if (result < 1) {
                return false;
            }
        } catch (SQLException e) {
            e.printStackTrace();
            return false;
        }
        return true;
    }
}
\end{lstlisting}

削除処理も入れてみた。

\begin{lstlisting}[caption=SelectEmployeeSample.java]
import java.util.List;
import model.Employee;
import dao.EmployeeDAO;

public class SelectEmployeeSample {
    public static void main( String[] args ) {

        EmployeeDAO empDAO = new EmployeeDAO();
        List<Employee> empList = empDAO.findAll();
        
        // 一覧
        for (Employee emp : empList) {
            System.out.println("ID:" + emp.getId());
            System.out.println("名前:" + emp.getName());
            System.out.println("年齢:" + String.valueOf(emp.getAge()) + "\n");
        }

        // 削除
        String id = "EMP003";
        if (empDAO.remove(id)) {
            System.out.println(id + "を削除しました。");
        }
    }
}
\end{lstlisting}

プロジェクトdaoの先頭でコンパイルをおこなう。そして classes フォルダに移動する。

\begin{tcolorbox}
\begin{verbatim}
chap13-mysql\dao> ant
chap13-mysql\dao> cd classes
\end{verbatim}
\end{tcolorbox}

サンプルデータをひとつ増やしておく。

\begin{tcolorbox}
 \begin{verbatim}
> mysql -u sa -p example
Password: **    (sa)

mysql> insert into employee values ('EMP003', '猿飛佐助', 25);
mysql> select * from employee;
  ( データの表示 )
 \end{verbatim}
\end{tcolorbox}


ここで実行する。

\begin{tcolorbox}
 \verb! classes> java -cp .;../lib/mysql-connector-java-8.0.21.jar SelectEmployeeSample !
\end{tcolorbox}

3つのデータが表示されたあと、``EMP003を削除しました。`` と表示されたら成功である。



\end{document}

%% 修正時刻: Sat May  2 15:10:04 2020


%% 修正時刻: Sat May  2 16:26:44 2020
