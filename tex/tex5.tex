\documentclass[dvipdfmx]{jsarticle}

\title{スッキリわかるサーブレット\&JSP入門 第13章をMySQLを使ってやってみる}
\author{Seiichi Nukayama}
\date{2020-07-26}
\usepackage{tcolorbox}
\usepackage{color}
\usepackage{listings, plistings}

% Java
\lstset{% 
  frame=single,
  backgroundcolor={\color[gray]{.9}},
  stringstyle={\ttfamily \color[rgb]{0,0,1}},
  commentstyle={\itshape \color[cmyk]{1,0,1,0}},
  identifierstyle={\ttfamily}, 
  keywordstyle={\ttfamily \color[cmyk]{0,1,0,0}},
  basicstyle={\ttfamily},
  breaklines=true,
  xleftmargin=0zw,
  xrightmargin=0zw,
  framerule=.2pt,
  columns=[l]{fullflexible},
  numbers=left,
  stepnumber=1,
  numberstyle={\scriptsize},
  numbersep=1em,
  language={Java},
  lineskip=-0.5zw,
  morecomment={[s][{\color[cmyk]{1,0,0,0}}]{/**}{*/}},
}
%\usepackage[dvipdfmx]{graphicx}
\usepackage{url}
\usepackage[dvipdfmx]{hyperref}
\usepackage{amsmath, amssymb}
\usepackage{itembkbx}
\usepackage{eclbkbox}	% required for `\breakbox' (yatex added)
\usepackage{enumerate}
\fboxrule=0.5pt
\parindent=1em
\begin{document}

%\anaumeと入力すると穴埋め解答欄が作れるようにしてる。\anaumesmallで小さめの穴埋めになる。
\newcounter{mycounter} % カウンターを作る
\setcounter{mycounter}{0} % カウンターを初期化
\newcommand{\anaume}[1][]{\refstepcounter{mycounter}{#1}{\boxed{\phantom{aa}\themycounter \phantom{aa}}}} %穴埋め問題の空欄作ってる。
\newcommand{\anaumesmall}[1][]{\refstepcounter{mycounter}{#1}{\boxed{\tiny{\phantom{a}\themycounter \phantom{a}}}}}%小さい版作ってる。色々改造できる。

%% 修正時刻: Tue May  5 10:19:29 2020


\section{どこつぶでデータベースを利用する}

\subsection{サンプルデータの作成}

MySQLでプログラムで使うためのサンプルデータを作る。

前回までで ``\textsf{sa}''というユーザーを作成しているので、そのユーザーに、
これから作成するデータベースについての権限を与えておく。

\begin{tcolorbox}
\begin{verbatim}
> mysql -u root -p
Password:       (なし)

mysql> grant all privileges on docoTsubu.* to 'sa'@'localhost';
mysql> quit;
\end{verbatim}
\end{tcolorbox}

次に新しいデータベース \textsf{docoTsubu} を作成する。

\begin{tcolorbox}
\begin{verbatim}
> mysql -u sa -p
Password: **      (sa)

mysql> create database docoTsubu default character set=utf8;
mysql> use docoTsubu;
mysql> create table mutter (
    -> id int primary key auto_increment,
    -> name varchar(100) not null,
    -> text varchar(255) not null
    -> ) default character set=utf8;
mysql> insert into mutter ( name, text ) values ( '湊', '今日は休みだ' ) ;
mysql> insert into mutter ( name, text ) values ( '綾部', 'いいな?' ) ;
mysql> select * from mutter;

+----+------+--------------+
| id | name | text         |
+----+------+--------------+
|  1 | 湊   | 今日は休みだ |
|  2 | 綾部 | いいな~     |
+----+------+--------------+
2 rows in set (0.00 sec)
\end{verbatim}
\end{tcolorbox}

insert文でデータを追加する時、\textsf{id} は \textsf{auto\_increment}
(自動追加)なので指定しなくてよい。


\subsection{プログラムの作成}

フォルダの構成は以下のとおり。

\begin{tcolorbox}
\begin{verbatim}
docoTsubuDao/
├── WEB-INF/
│   ├── classes/
│   └── jsp/
└── src/
    ├── dao/
    ├── model/
    └── servlet/
\end{verbatim}
\end{tcolorbox}

''sukkiri''フォルダの中に ''docoTsubuDao''フォルダを作成し、その中に上記のフォルダを
作成する。


p399からのコードを入力していく。

これは本のコードのままである。

\begin{lstlisting}[caption=model/Mutter.java]
 package model;

 import java.io.Serializable;

 public class Mutter implements Serializable {
   private int id;
   private String userName;
   private String text;
  
   public Mutter () {}
  
   public Mutter (String userName, String text) {
     this.userName = userName;
     this.text = text;
   }
  
   public Mutter (int id, String userName, String text) {
     this.id = id;
     this.userName = userName;
     this.text = text;
   }
  
   public int getId () { return id; }
   public String getUserName () { return userName; }
   public STring getText () { return text; }
 }
\end{lstlisting}

次のコードは、jdbsドライバを読み込んでいる部分が違うだけで、あとは同じ。

ただ、プログラムがちゃんと動いているかをチェックするためのコードを入れておいた。

\begin{lstlisting}[caption=dao/MutterDAO.java]
 package dao;

 import java.sql.Connection;
 import java.sql.DriverManager;
 import java.sql.PreparedStatement;
 import java.sql.ResultSet;
 import java.sql.SQLException;
 import java.util.ArrayList;
 import java.util.List;
 import model.Mutter;
 
 public class MutterDAO {
   // info about database
   private final String JDBC_URL =
       "jdbc:mysql://localhost:3306/docoTsubu?serverTimezone=JST";
   private final String DB_USER = "sa";
   private final String DB_PASS = "sa";

   public List<Mutter> findAll () {
     List<Mutter> mutterList = new ArrayList <> ();

     // connect
     try (Connection conn =
         DriverManager.getConnection( JDBC_URL, DB_USER, DB_PASS )) {
       
       String sql =
           "select id, name, text from mutter order by id desc";
       PreparedStatement pStmt = conn.prepareStatement( sql );

       ResultSet rs = pStmt.executeQuery();

       while (rs.next()) {
         int id = rs.getInt("id");
         String userName = rs.getString("name");
         String text = rs.getString("text");
         System.out.println("CHECK: " + userName);        // チェック用
         Mutter mutter = new Mutter( id, userName, text );
         mutterList.add(mutter);
       }
     } catch (SQLException e) {
       e.printStackTrace();
       System.out.println("エラーでっせ!");
       return null;
     }
     return mutterList;
   }

   public boolean create( Mutter mutter ) {
     try (Connection conn = 
         DriverManager.getConnection(
               JDBC_URL, DB_USER, DB_PASS)) {
       String sql = "insert into mutter( name, text ) values (?, ?)";
       PreparedStatement pStmt = conn.prepareStatement( sql );
       pStmt.setString( 1, mutter.getUserName() );
       pStmt.setString( 2, mutter.getText() );

       // resultには追加された行数が入る
       int result = pStmt.executeUpdate();
       if (result != 1) {
         return false;
       }
     } catch (SQLException e) {
         e.printStackTrace();
         System.out.println("データの追加失敗");
       return false;
     }
     return true;
   }
 }
 \end{lstlisting}

\begin{lstlisting}[caption=servlet/Main.java]
 package servlet;
 
 import java.io.IOException;
 import java.util.ArrayList;
 import java.util.List;
 
 import javax.servlet.RequestDispatcher;
 import javax.servlet.ServletContext;
 import javax.servlet.ServletException;
 import javax.servlet.annotation.WebServlet;
 import javax.servlet.http.HttpServlet;
 import javax.servlet.http.HttpServletRequest;
 import javax.servlet.http.HttpServletResponse;
 import javax.servlet.http.HttpSession;
 
 import model.GetMutterListLogic;
 import model.Mutter;
 import model.PostMutterLogic;
 import model.User;
 
 @WebServlet("/Main")
 public class Main extends HttpServlet {
   private static final long serialVersionUID = 1L;

   protected void doGet (HttpServletRequest request,
                         HttpServletResponse response)
     throws ServletException, IOException {

     // つぶやきリストを取得して、リクエストスコープに保存
     GetMutterListLogic getMutterListLogic =
         new GetMutterListLogic();
     List<Mutter> mutterList = getMutterListLogic.execute();
     request.setAttribute("mutterList", mutterList);
     if (mutterList.size() > 0) {
         System.out.println("空ではない!");
     }

     // セッションスコープからユーザー情報(loginUser)を取得
     HttpSession session = request.getSession();
     User loginUser = (User) session.getAttribute("loginUser");

     if (loginUser == null) {
         // ユーザー情報が空なら新規作成画面へリダイレクト
         response.sendRedirect("/docoTsubuDao");
     } else {
         RequestDispatcher dispatcher =
             request.getRequestDispatcher("/WEB-INF/jsp/main.jsp");
         dispatcher.forward( request, response );
     }
   }
 
   @SuppressWarnings("unchecked")
   public static List<Mutter> automaticCast( Object src ) {
       List<Mutter> castedObject = (List<Mutter>) src;
       return castedObject;
   }
 
   protected void doPost (HttpServletRequest request,
                          HttpServletResponse response)
     throws ServletException, IOException {

     request.setCharacterEncoding("UTF-8");
     String text = request.getParameter("text");

     // check
     if (text != null && text.length() != 0) {

       // User info
       HttpSession session = request.getSession();
       User loginUser = (User) session.getAttribute("loginUser");

       // added Tsubuyaki List
       Mutter mutter = new Mutter( loginUser.getName(), text );
       PostMutterLogic postMutterLogic = new PostMutterLogic();
       postMutterLogic.execute(mutter);
         
     } else {
       // textが 空だった場合
       request.setAttribute("errorMsg", "つぶやきが入力されていません");
     }

     GetMutterListLogic getMutterListLogic =
         new GetMutterListLogic ();
     List<Mutter> mutterList = getMutterListLogic.execute();
     request.setAttribute("mutterList", mutterList);

     // forward to main
     RequestDispatcher dispatcher =
         request.getRequestDispatcher("/WEB-INF/jsp/main.jsp");
     dispatcher.forward( request, response );
   }
 }
\end{lstlisting}

\begin{lstlisting}[caption=model/GetMutterListLogic.java]
 package model;

import java.util.List;
import dao.MutterDAO;

public class GetMutterListLogic {
	public List<Mutter> execute () {
		MutterDAO dao = new MutterDAO ();
		List<Mutter> mutterList = dao.findAll ();
		return mutterList;
	}
}
\end{lstlisting}


以下は、第11章で作成したものをそのまま使う。


\begin{lstlisting}[caption=model/LoginLogic.java]
 package model;

public class LoginLogic {
    public boolean execute (User user) {
        if (user.getPass().equals("1234")) { return true; }
        return false;
    }
}
\end{lstlisting}

\begin{lstlisting}[caption=model/PostMutterLogic.java]
 package model;

import dao.MutterDAO;

public class PostMutterLogic {
    public void execute (Mutter mutter) {
        MutterDAO dao = new MutterDAO();
        dao.create( mutter );
    }
}
\end{lstlisting}

\begin{lstlisting}[caption=model/User.java]
 package model;

import java.io.Serializable;

public class User implements Serializable {
    private String name;
    private String pass;

    public User () {}
    public User (String name, String pass) {
        this.name = name;
        this.pass = pass;
    }

    public String getName () { return name; }
    public String getPass () { return pass; }
}
\end{lstlisting}

\begin{lstlisting}[caption=servlet/Login.java]
 package servlet;

import java.io.IOException;

import javax.servlet.RequestDispatcher;
import javax.servlet.ServletException;
import javax.servlet.annotation.WebServlet;
import javax.servlet.http.HttpServlet;
import javax.servlet.http.HttpServletRequest;
import javax.servlet.http.HttpServletResponse;
import javax.servlet.http.HttpSession;

import model.LoginLogic;
import model.User;

@WebServlet("/Login")
public class Login extends HttpServlet {
    private static final long serialVersionUID = 1L;

    protected void doPost (HttpServletRequest request,
                           HttpServletResponse response)
        throws ServletException, IOException {

        request.setCharacterEncoding("UTF-8");
        String name = request.getParameter("name");
        String pass = request.getParameter("pass");

        User user = new User (name, pass);

        /**
         * ログイン処理
         * LoginLogic.execute
         * user.getPass() == 初期パスワード ==> true
         *      ..        !=       ..       ==> false
         */
        LoginLogic loginLogic = new LoginLogic();
        boolean isLogin = loginLogic.execute (user);

        if (isLogin) {
            HttpSession session = request.getSession();
            session.setAttribute("loginUser", user);
        }

        RequestDispatcher dispatcher =
            request.getRequestDispatcher("/WEB-INF/jsp/loginResult.jsp");
        dispatcher.forward( request, response );
    }
}
\end{lstlisting}


\begin{lstlisting}[caption=servlet/Logout.java]
 package servlet;

import java.io.IOException;

import javax.servlet.RequestDispatcher;
import javax.servlet.ServletException;
import javax.servlet.annotation.WebServlet;
import javax.servlet.http.HttpServlet;
import javax.servlet.http.HttpServletRequest;
import javax.servlet.http.HttpServletResponse;
import javax.servlet.http.HttpSession;

@WebServlet("/Logout")
public class Logout extends HttpServlet {
    private static final long serialVersionUID = 1L;

    protected void doGet( HttpServletRequest request,
                          HttpServletResponse response )
        throws ServletException, IOException {

        HttpSession session = request.getSession();
        // セッションスコープの破棄
        session.invalidate();

        RequestDispatcher dispatcher =
            request.getRequestDispatcher("/WEB-INF/jsp/logout.jsp");
        dispatcher.forward( request, response );
    }
}
\end{lstlisting}

\begin{lstlisting}[caption=common.jsp]
 <%@ page language="java" contentType="text/html; charset=UTF-8"
         pageEncoding="UTF-8" %>
<%@ page import="java.util.Date, java.text.SimpleDateFormat" %>
<% String name = "猿飛佐助"; %>
\end{lstlisting}


\begin{lstlisting}[caption=footer.jsp]
 <%@ page language="java" contentType="text/html; charset=UTF-8"
         pageEncoding="UTF-8" %>
<p>Copyright &copy; どこつぶ制作委員会 All Rights Reserved.</p>
\end{lstlisting}


\begin{lstlisting}[caption=index.jsp]
 <%@ page language="java" contentType="text/html; charset=UTF-8"
         pageEncoding="UTF-8" %>

<%@ include file="/common.jsp" %>
<%
Date date = new Date();
SimpleDateFormat sdf = new SimpleDateFormat("YYYY年MM月dd日");
String today = sdf.format(date);
%>
<!doctype html>
<html lang="ja">
  <head>
    <meta charset="utf-8"/>
    <title>どこつぶ</title>
  </head>
  <body>
    <h1>どこつぶへようこそ</h1>
    <time><%= today %></time>
    <p>管理人:<%= name %></p>
    <form action="/docoTsubuDao/Login" method="post">
      ユーザー名:<input type="text" name="name"/><br/>
      パスワード:<input type="password" name="pass"/><br/>
      <input type="submit" value="ログイン"/>
    </form>
    <jsp:include page="/footer.jsp" /> 
  </body>
</html>
\end{lstlisting}

\subsection{ビルド}

build.xml を以下のようにする。

\begin{lstlisting}
 <?xml version="1.0" ?>
<project name="docoTsubu_dao" default="compile" basedir=".">
  <target name="compile">
      <javac includeAntRuntime="false"
             encoding="UTF-8"
             srcdir="./src"
             destdir="./WEB-INF/classes"
             classpath="C:\pleiades\tomcat\9\lib\servlet-api.jar;
                        C:\pleiades\tomcat\9\lib\jsp-api.jar;
                        C:\pleiades\tomcat\9\lib\mysql-connector-java-8.0.21.jar"
             />
  </target>
</project>
\end{lstlisting}

C:/pleiades/tomcat/9/conf/Catalina/localhost のフォルダに以下のように
 ``\textsf{docoTsubuDao.xml}''を作る。

\begin{lstlisting}[caption=docoTsubuDao.xml]
<?xml version='1.0' encoding='utf-8'?>
<Context path="/docoTsubuDao"
    docBase="C:\Users\user\sukkiri\chap13-mysql\docoTsubuDao" />
\end{lstlisting}

\include{end}

%% 修正時刻: Tue Jul 28 08:26:26 2020
