\documentclass[dvipdfmx]{jsarticle}

\title{スッキリわかるサーブレット\&JSP入門 第13章をMySQLを使ってやってみる}
\author{Seiichi Nukayama}
\date{2020-07-26}
\usepackage{tcolorbox}
\usepackage{color}
\usepackage{listings, plistings}

% Java
\lstset{% 
  frame=single,
  backgroundcolor={\color[gray]{.9}},
  stringstyle={\ttfamily \color[rgb]{0,0,1}},
  commentstyle={\itshape \color[cmyk]{1,0,1,0}},
  identifierstyle={\ttfamily}, 
  keywordstyle={\ttfamily \color[cmyk]{0,1,0,0}},
  basicstyle={\ttfamily},
  breaklines=true,
  xleftmargin=0zw,
  xrightmargin=0zw,
  framerule=.2pt,
  columns=[l]{fullflexible},
  numbers=left,
  stepnumber=1,
  numberstyle={\scriptsize},
  numbersep=1em,
  language={Java},
  lineskip=-0.5zw,
  morecomment={[s][{\color[cmyk]{1,0,0,0}}]{/**}{*/}},
}
%\usepackage[dvipdfmx]{graphicx}
\usepackage{url}
\usepackage[dvipdfmx]{hyperref}
\usepackage{amsmath, amssymb}
\usepackage{itembkbx}
\usepackage{eclbkbox}	% required for `\breakbox' (yatex added)
\usepackage{enumerate}
\fboxrule=0.5pt
\parindent=1em
\begin{document}

%\anaumeと入力すると穴埋め解答欄が作れるようにしてる。\anaumesmallで小さめの穴埋めになる。
\newcounter{mycounter} % カウンターを作る
\setcounter{mycounter}{0} % カウンターを初期化
\newcommand{\anaume}[1][]{\refstepcounter{mycounter}{#1}{\boxed{\phantom{aa}\themycounter \phantom{aa}}}} %穴埋め問題の空欄作ってる。
\newcommand{\anaumesmall}[1][]{\refstepcounter{mycounter}{#1}{\boxed{\tiny{\phantom{a}\themycounter \phantom{a}}}}}%小さい版作ってる。色々改造できる。

%% 修正時刻: Tue May  5 10:19:29 2020


\section{JDBCドライバーをインストールする}

JDBCドライバーをWindowsにインストールするのは、ネットで調べてもちょっとわかりにくいかもしれない。
情報が古いこともある。また、Oracle がWebサイトをけっこう頻繁に模様替えしてるのもある。

\subsection{ダウンロード}

まず、ここにいく。

\href{https://www.mysql.com/jp/products/connector/}{https://www.mysql.com/jp/products/connector/}

JDBC Driver for MySQL(Connector/J) の行の \textgt{ダウンロード} をクリックする。


\href{MySQL Community Downloads}{https://dev.mysql.com/downloads/connector/j/} のページに遷移するので、ここのところからインストールする。

``\textsf{Connector/J 8.0.27}''と書かれているところの
``\textsf{Select Operating System...}'' から、\textsf{Windows} を選択する。

Windows は、その下の ``\textsf{MySQL Installer for Windows}'' をインストールして、それを使ってインストールすることになる。

大きなバナーをクリックするか、その下の
``\textsf{Windows (x86, 32 \& 64-bit), MySQL Installer MSI}'' の横の
 ``\textsf{Go to Download Page \textgreater}'' をクリックする。

 \href{https://dev.mysql.com/downloads/connector/j/}{MySQL Community Downloas/MySQL Installer 8.0.27}
 というページが開き、少し下に ``\textsf{MySQL Installer 8.0.27}'' というコラムがある。
 ``\textsf{Generally Available (GA) Releases}'' とある。

その中に二つの \textsf{Download} ボタンがある。

上は、``\textsf{Windows(x86, 32-bit), MSI Installer 8.0.27 2。3M}'' とある。
ダウンロードファイルは、``\textsf{mysql-installer-web-community-8.0.27.1.msi}''

下は、``\textsf{Windows(x86, 32-bit), MSI Installer 8.0.27 470.2M}'' とある。
ダウンロードファイルは、``\textsf{mysql-installer-community-8.0.27.1.msi}''

結論から言うと、下の方をダウンロードしたほうがうまくいった。
でも、どちらを選んでもできることは同じようである。


\subsection{インストール}

上記ファイルをクリックすると、``\textsf{Login Now or Sign Up for a free account}'' とあって、
ユーザー認証/登録を促される。別に登録してもかまわない。

ここでは、左下の ``\textsf{No thanks, just start my download}'' を選択する。


ダウンロードフォルダに ``\textsf{mysql-installer-community-8.0.27.1.msi}'' がダウンロードされる。
このファイルをダブルクリックして \textsf{インストール} を始める。

「このアプリがデバイスに変更を加えることを許可しますか?」と聞かれるので、「はい」をクリックする。

``\textsf{Choosing a Setup Type}'' のダイアログが開く。ここではインストールするものを選べる。
通常は、``\textsf{Developer Default}'' を選択するのだが、今回は、「JDBCドライバー」のみ必要なので、
一番下の``\textsf{Custom}'' を選択して、``\textsf{Next}''。

``\textsf{Select Products and Features}'' のダイアログが開く。
``\textsf{MySQL Connectors}'' の項目の「+」をクリックする。

``\textsf{Connector/J}'' - ``\textsf{Connector/J 8.0}'' - ``\textsf{Connector/J 8.0.27-X86}''
とクリックすると、右側の矢印が緑色に変わるので、その矢印をクリックする。

すると、右側に ``\textsf{Connector/J 8.0.27-X86}'' が表示される。
``\textsf{Next}'' をクリックする。

``\textsf{Installation}'' のダイアログが開く。
``\textsf{Connector/J 8.0.27}'' が表示されており、``\textsf{Ready to Install}'' となっている。
下の ``\textsf{Execute}'' ボタンをクリックする。

``\textsf{Complete}'' となる。 ``\textsf{Next}'' をクリックする。

``\textsf{Installation Complete}'' となり、 ``\textsf{Copy Log to Clipboard}'' ボタンを押しておく。
``\textsf{Finish}'' ボタンをクリックして終了。

コピーしておいたログをメモ帳などで見てみると、どこにインストールされたかわかるので、のちに役立つ。このログは ``\textsf{connector-j.log}'' とでも名前をつけて保存しておく。

\subsection{jdbcドライバのインストール}

このドライバファイルは、Tomcatがインストールされているディレクトリの \textsf{lib} ディレクトリに置く。


\subsection{補足}

MySQL Connector/J 8.0については、以下のように書かれてある。

MySQL Connector/J 8.0 is highly recommended for use with MySQL Server 8.0, 5.7, 5.6, and 5.5. Please upgrade to MySQL Connector/J 8.0.
(MySQL Connector / J 8.0は、MySQL Server 8.0、5.7、5.6、および5.5で使用することを強くお勧めします。 MySQL Connector / J 8.0にアップグレードしてください。by Google翻訳)

MySQL5.7などでも使えるようである。





\include{end}

%% 修正時刻: Thu Dec 23 21:09:08 2021


